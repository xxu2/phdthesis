\chapter{Some Tables and Figures}

\begin{table}[h]
  \centering
  \begin{tabular}{ll}\toprule
    First & Last \\ \midrule
    Ned & Hummel \\
    Ned & Hummel \\
    Ned & Hummel \\ \bottomrule
  \end{tabular}
  \caption{Arma virumque cano, Troiae qui primus ab oris Italiam, fato profugus,
Laviniaque venit litora, multum ille et terris iactatus et alto vi
superum saevae memorem Iunonis ob iram}
  \label{tab:tabular}
\end{table}

\begin{table}[h]
  \centering

  \begin{compactitem}[\checkmark]
    \item Foo
    \item Foo
    \item Foo
    \end{compactitem}

  \caption{Arma virumque cano, Troiae qui primus ab oris Italiam, fato profugus,
Laviniaque venit litora, multum ille et terris iactatus et alto vi
superum saevae memorem Iunonis ob iram}
  \label{tab:list}
\end{table}

\begin{figure}[h]
  \centering
  \includegraphics[width=3in]{figures/unl}
  \caption{Arma virumque cano, Troiae qui primus ab oris Italiam, fato profugus,
Laviniaque venit litora, multum ille et terris iactatus et alto vi
superum saevae memorem Iunonis ob iram}
  \label{fig:test}
\end{figure}

\chapter{Some Math}\label{chap:math}

This is a triviality, but we include it for completeness.
\begin{equation}
\int_0^\infty f(x) \, dx =
\begin{cases} 1 & \mbox{if $f=\delta$,} \\
0 & \mbox{if $f=0$.} \end{cases}
\end{equation}

Here is an aligned set of equations.
\begin{align}
f(x) &= f(x) \cdot 1 \\
     &= f(x) \cdot (2-1)\label{eq:fun}\\
     &= f(x)
\end{align}

The clever step is~\eqref{eq:fun}.
