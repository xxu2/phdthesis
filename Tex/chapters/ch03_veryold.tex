% Chapter 3
\chapter{Adjoint Inversion of Dust Emissions}

 \textit{Overview} \hspace{0.2cm}

\section{The Adjoint Inversion Framework}

 The adjoint of the GEOS-Chem model was developed specifically 
 for inverse modeling of aerosol (or their precursors) 
 and gas emissions \citep{henze07,henze09}, 
 and it is continuously improved and maintained 
 by the GEOS-Chem Adjoint and Data Assimilation 
 Working Group and its users 
 (\url{http://wiki.seas.harvard.edu/geos-chem/index.php/GEOS-Chem_Adjoint}). 
 The strength of the adjoint model is 
 its ability to efficiently calculate model sensitivities 
 with respect to large sets of model parameters, 
 such as aerosol emissions at each grid box. 
 These sensitivities can serve as the gradients needed for 
 inverse modeling of aerosol emissions. 
 Recent studies have used the GEOS-Chem adjoint with 
 satellite observations to constrain sources of species 
 such as \ce{CO} \citep{kopacz09,kopacz10,jiang11}, 
 \ce{CH4} \citep{,wecht12}, and \ce{O3} \citep{parrington12}
 to diagnose source regions for long-range transport \citep{henze09,kopacz11}, 
 and to provide guidance on future geostationary observations 
 of surface air quality \citep{zoogman11}.

 \subsection{GEOS-Chem adjoint modeling}

 The GES-Chem adjoint model was specifically developed for \ldots \citep{henze07,henze09}.
 It has been widely used to \ldots [references\ldots]. 

 It caluclates the adjoint, or the transpose of Jacobian matrix of receptor with respect 
 to the state vector, following the \ldots

 Based on the infrastructure of GEOS-Chem, we need to develope (1) an observation operator 
 that maps the aerosol concentration into the observation space, (2) the capacity of 
 calculating the adjoint with respect to dust emission flux, and (3) the capacity of 
 calculating the adjoint with respect to parameters in dust emission scheme. 

\section{Necessary Implementations}

 \subsection{AOD observation operator}

  Inversion of dust emissions from AOD observations requires 
  an observation operator that relates AOD to emission. 
  Such observation operator of this study comprises 
  a GEOS-Chem forward model that simulates aerosol concentrations 
  with input of emissions and an AOD operator that computes AOD 
  from aerosol concentrations. 
  AOD at any wavelength is calculated from the sum of AODs 
  of each component assuming external mixing state of aerosol
  particles:
  \begin{equation} \label{eq:aod} 
   \tau_\text{A} = \displaystyle \sum_{i=1}^{n_\text{spec}} 
                   {\beta_\text{ext}}_i c_i
                   \frac{3c_i {Q_\text{ext}}_i}{4 \rho_i, {r_\text{eff}}_i}
  \end{equation}
  where $n_\text{spec}$ is the number of aerosol components, 
  $c_i$ is the aerosol mass concentration of the $i$th component, 
  and ${\beta_\text{ext}}_i$ is the mass extinction efficiency. 
  ${\beta_\text{ext}}_i$ is related to the 

 
 \subsection{Adjoint for Dust Emissions}
 \subsection{Adjoint for Dust Flux Parameterization}

   In simple, The dust emission flux considering subgrid wind speeds in equation 
   (\ref{eq:Edp}) can be writen
   \begin{equation}
   E_{d,j} = C_j S^\prime \int^{u_{*u}}_{u_{*t}} Q_s(u_*,u_{*t},b) p(u_*) d u_* \mbox{,}
   \end{equation}
   where $S^\prime = S \alpha$, and $C_j=A_m \displaystyle \sum_{i=1}^3 M_{i,j}$. We combine
   the erodibility $S$ and sandblasting factor $\alpha$, because both of them not only are 
   related to the soil texture but also describe the strength efficiency of dust emission. 
   Given the the state of land surface and the properties of surface soil, the dust emission
   is a function of $S^\prime$, $b$, and $u_{*t}$. 

   Here we implement the adjoint calculation for three parameters, i.e., 
   $S^\prime$, $b$, and $u_{*t}$. This implementation reqiures the partial
   derivatives of $E_{d,j}$ with respect to these parameters (when 
   $u_* \geq u_{*t}$): 
   \begingroup
   \allowdisplaybreaks
   \begin{align}
     \frac{\partial E_{d,j}}{\partial S^\prime} 
       &= \frac{E_{d,j}}{S^\prime}\mbox{,} \\
     \frac{\partial E_{d,j}}{\partial b} 
       &= C_j S^\prime \int^{u_{*u}}_{u_{*t}} 
          \frac{\partial Q_s}{\partial b} p(u_*) d u_* \mbox{,} \label{eq:dEdb}\\
     \frac{\partial E_{d,j}}{\partial u_{*t}}
       & = C_j S^\prime \int^{u_{*u}}_{u_{*t}} 
          \frac{\partial Q_s}{\partial u_{*t}} p(u_*) d u_* \mbox{.} \label{eq:dEdut}
   \end{align}
   \endgroup
   These gradients of $Q_s$ in equations (\ref{eq:dEdb} and \ref{eq:dEdut}) can be 
   calculated by
   \begingroup
   \allowdisplaybreaks
   \begin{align}
   \frac{\partial Q_s}{\partial b} &= Q_s(u_*,u_{*t},b) \ln{u_*} \\
   \frac{\partial Q_s}{\partial u_{*t}} &=\frac{c_s \rho}{g} u_*^b
         \left[\frac{1}{u_*} - \frac{2u_{*t}}{u_*^2} 
         - \frac{3u_{*t}^2}{u_*^3}\right] 
   \end{align}
   \endgroup
