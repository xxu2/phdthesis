% Chapter one

\chapter{Introduction}

\section{Background and Motivation}

Atmospheric aerosols play a crucial role in the global
climate change. They affect earth energy budget directly
by scattering and absorbing solar and terrestrial radiation,
and indirectly through altering the cloud formation,
lifetime, and radiative properties [Haywood and Boucher,
2000; Ramanathan et al., 2001]. However, quantification of
these effects in the current climate models is fraught with
uncertainties. The global average of aerosol effective
radiative forcing (ERF) were estimated to range from -0.1
to -1.9 Wm2 with the best estimate of -0.9 Wm2 [Boucher et
al., 2013], indicating that the cooling effects of aerosol
might counteract the warming effects of 1.820.19 Wm2 caused
by the increase of carbon dioxide since the industrial
revolution [Myhre et al., 2013]. The climate effects of
aerosol particles depend on their geographical distribution,
optical properties, and efficiency as cloud condensation
nuclei (CCN). Key quantities pertain to the aerosol optical
and cloud-forming properties include particle size
distribution (PSD), chemical composition, mixing state, and
morphology [Boucher et al., 2013]. While the daily aerosol
optical depth (AOD) can be well measured from current
satellite and ground-based remote sensing instrumentations
[e.g., Holben et al., 1998; Kaufman et al., 2002], the
accurate quantification of aerosol ERF is in no small part
hindered by our limited knowledge about the aerosol PSD and
refractive index (describing chemical composition and
mixing state). To fully understand the role of aerosol
particles in the global climate change, further development
in observations along with retrieval algorithms for these
aerosol microphysical properties from different platforms
are thus highly needed [Mishchenko et al., 2004], and
the focus of this two-part series study is the
characterization of aerosol properties from ground-based
passive remote sensing \citep{henze07}.

  Dust acts as medium in iron cycle on the earth \citep{jickells05}.

  The long-distance transport of bacteria together with dust
  has been discussed in several studies as a mechanism for
  the global dispersion of microbial species,
  with the potential to impact ecosystems and public health
  \citep{griffin01,burrows09}

 In the fllowing paragraphs of this section, 
 I briefly describe the impacts of dust aerosols on the earth system, 
 the basic physics of dust emissions, 
 and the current state of satellite remote sensing of mineral dust. 
 Subsequently, I present the specific scientific objectives and the 
 organization of this thesis.  

\subsection{Top-down versus bottom-up}

 Current estimates of aerosol emissions are largely based on the “bottom-up” method 
 that integrates diverse information such as fuel consumption in various industries 
 and corresponding measurements of emission rates for different species \citep{streets03},
 economic growth, and the statistics of land use and fire-burned areas 
 [van der Werf et al., 2006]. While significant progress has been made [Streets et al., 2006], 
 the “bottom-up” approach has a number of limitations.  
 First, the emission inventory usually has a temporal lag of at least 2 to 3 years, 
 as time is needed to aggregate information from different sources 
 and format them into the emission inventories that are suitable for use in climate models. 
 Second, the temporal resolution of the current emission inventory 
 is usually on monthly to annual scale, which is not sufficient 
 to characterize the daily or diurnal variation of emissions; 
 the aerosol impact on radiative transfer and the variation of cloud properties, 
 however, is often strongly dependent on the time of the day [Wang et al., 2006]. 
 Third, the spatial resolutions of the bottom-up emission inventories are usually 
 limited by the availability of the ground-based observations, 
 which often lack the spatial coverage for estimating emission 
 in a uniformly fine resolution for regional modeling of aerosol transport. 
 Finally, bottom-up emission inventories may miss important emission sources 
 that are not well documented including emissions from wild fires, 
 volcanic eruptions, and agricultural activities. 
 All these limitations are amplified over the East Asia region 
 because the economic growth in China is so rapid that information needed 
 for bottom-up approach cannot be timely and reliably documented. 

 To complement information from bottom-up emissions, remote sensing is increasingly used 
 to better quantify aerosol distributions. 
 The satellite observations and/or products can provide information important 
 for the bottom-up estimate of emissions. 
 Examples include the fire products from MODIS, ASTER, and AVHRR sensors 
 that are widely used for characterizing the biomass burning emissions 
 [Borrego et al., 2008; van der Werf et al., 2006; 2010; Reid et al., 2009]. 
 Alternatively, the satellite observed tracer abundance could be used 
 to constrain bottom-up estimates of aerosol emissions through the inverse modeling; 
 such method is referred to as a ‘top-down’ constraint. 
 Although satellite-based aerosol retrievals have less precision than in situ measurements, 
 studies have shown that they are able to quantify the atmospheric aerosol loading 
 and temporal variations with good agreement and expected accuracy 
 to the ground-based observations [Levy et al., 2010; Remer et al., 2005]. 
 Furthermore, the satellite-based aerosol data, in contrast to the ground-based ones, 
 have much higher temporal resolution across the globe. For instance, the MODIS sensor, 
 aboard on NASA’s both Terra and Aqua satellites, has a surface footprint size of 
 about 1 km at nadir and needs only 1 to 2 days to achieve global coverage. 
 In addition, the joint retrieval of aerosols from diverse satellite sensors 
 enhances the accuracy of satellite aerosol products [Sinyuk et al., 2008], 
 the potential of which have also been shown in the air quality monitoring 
 [Liu et al., 2005; Wang et al., 2010]. 

 Different top-down techniques have been developed to optimally estimates 
 the emissions from satellite observations, which include but are not 
 limited to the following: 
 \begin{itemize}
 \item (a) the use of a scaling factor that is the ratio of observed tracer abundances to 
 the CTM simulated counterparts [e.g. Lee et al., 2011; Martin et al., 2003; Wang et al., 2006];
 \item (b) the use of the local sensitivity of change of tracer concentration 
            to the change of emission [e.g. Lamsal et al., 2011; Walker et al., 2010]; 
 \item (c) the analytical Bayesian inversion method [e.g. Heald et al., 2004]; 
 \item (d) the adjoint of CTM [e.g. Müller and Stavrakou, 2005; 
           Henze et al., 2007; 2009; Dubovik et al., 2008; Kopacz et al., 2009; 2010; 
           Wang et al., 2012].  
 \end{itemize} 
 The first two methods are similar; 
 both assume a linear relationship between model simulated aerosol abundances and emissions. 
 The analytical method is exact but computationally expensive 
 and thus can only constrain emission in the domain-wise 
 or over coarse spatial resolution [Kopacz et al., 2009]. 
 In contrast to the first three approaches, the adjoint approach is designed 
 for exploiting the high-density of observations to constrain emission 
 with high resolution [Kopacz et al., 2009], as it is able to efficiently 
 calculate gradients of the overall mismatch between observations 
 and model estimates with respect to large sets of parameters 
 (i.e., emissions resolved at each grid box) [Henze et al., 2007].

 Several studies have successfully analyzed sources of 
 traces gases using the top-down methods, including \ce{CO} sources 
 from MOPITT sensor over the Asia [Heald et al., 2004; Kopacz et al., 2009;] 
 and over the globe [e.g. Stavrakou and Müller, 2006; Kopacz et al. 2010], 
 \ce{CO2} surface flux from the TES sensor [Nassar et al., 2011], 
 \ce{NOx} emissions from space-based column \ce{NO2} by several satellite sensors 
 [Lamsal et al., 2011; Lin et al., 2010; Martin et al., 2003; Müller and Stavrakou, 2005], 
 and \ce{SO2} from SCIAMACHY and OMI sensors [Lee et al., 2011], etc. 
 However, not all emissions of trace gases can be fully constrained with their 
 satellite-based counterpart products, because some trace gases (e.g. \ce{SO2}) 
 can react with other gases (e.g. \ce{NH3}), to form either liquid or solid aerosols 
 (e.g. \ce{(NH4)2SO4}). As a result, using measurements of trace gases alone 
 can only provide partial constraints on the emission of the corresponding trace gases. 

\subsection{Satellite Observations of Aerosol}

\begin{table}[h]
  \centering
  \small
  \caption{List of satellite sensors with measurement specifications 
  relevant for operation retrieval of aerosol properties.}
  \label{tab:satellites}
  %\begin{tabular}{p{0.09\textwidth}p{0.3\textwidth}p{0.3\textwidth}p{0.2\textwidth}}\toprule
  %  Sensor & Measurement Specification  & Retrieved Aerosol Property & References \\ \midrule
  %  AVHRR\textsuperscript{a}  & 15 bands at single view angle & AOD & Dubuisson et al [2009] \\
  %  CALIOP & Layer backscattering and depolarization ratio & AOD profile & Winker et al [2010] \\
  %  MERIS  & 15 bands at single view angle & AOD & Dubuisson et al [2009] \\ \bottomrule
  %\end{tabular}
\end{table}

\section{Main Goals of This Work}

 Based on the preceding discussions, this work aims at improved estimates of 
 global dust emissions through adjoint integration of AOD retrievals 
 from multiple satellite platforms (MODIS and MISR) with a CTM (GEOS-Chem). 
 The overall goal is to conduct the satellite-based global model estimates 
 of atmospheric dust distribution, and thereby advance the understanding of 
 the impacts of atmospheric mineral dust on climate change and air quality. 
 To accomplish this goal, this work pursues the following specific objectives:
 \begin{itemize}
 \item Develop a top-down numerical inversion scheme for constraining global dust emissions with a combined use of multi-platform AOD products and CTM adjoint, which also includes the sensitivity and error budget analysis for the optimization.
 \item Apply the inversion scheme developed in step 1 for a one year (i.e. 2008) of dust emissions with level 3 quality-controlled MODIS DB and MISR AOD products.
 \item A long-term (from 2001 to 2010) analysis of dust emissions will follow, along with studies on the seasonal and inter-annual variability of dust emissions, loadings, and direct radiative effects.
 \item Wherever possible, ground-based and field data will be used to validate and analyze the uncertainties of the inversion results.
 \end{itemize}
Although the adjoint optimization technique we use is similar to that in Dubovik et
al. [2008] and Yumimoto et al. [2007], this study differs from the those previous studies in that: (a) Multi-platform AOD products utilized to optimize dust emissions can provide tremendous dust information in fine spatial and temporal scales; (b) This study uses the satellite AOD retrievals only over and near dust source regions where dust has been transported a short distance with minimal influence of precipitation and anthropogenic
aerosols; (c) Optimization of the long-term dust emissions is conducted for every grid box as a function of time (e.g., on the weekly or month scale). Although the criteria for separation of a natural and anthropogenic dust source is not clear and sometime controversial in the literature [Denman et al., 2007], especially when considering the climatic feedback on dust emissions [Zhang et al., 2002], we believe that satellite-based optimization of global dust emissions in the last decade could improve our modeling of dust radiative forcing and potentially illuminate anthropogenic components of dust sources and loadings, currently estimated at 0-20\% though values as large as 50\% has been postulated [Ginoux et al., 2011; Tegen et al., 1996; 2004; Mahowald et al., 2004,].

\section{Organization of This Dissertation}

 We describe the GEOS-Chem simulation of mineral dust in Chapter 2 with emphersizing the 
 physical parameterization of dust sources, after which we present the implements for the 
 AOD observation operator and the adjoint capacity of dust emission within the GEOS-Chem
 ajoint model in Chapter 3. In chapter 4, we present a case study on optimizing the dust 
 emission estimates from the satellite (MODIS) radiances over the eastern Asia, 
 in which we also attempt to 
 simutaneously constrain the anthropogenic emissions of the \ce{SO2}, \ce{NO2}, \ce{NH3}, 
 and carboneous aerosols tegether with the dust aerosols. In chapter 5, we optimize the 
 dust source parameterization from multi-satellite AOD products, particularly in improving 
 the the estimates of soil erodibility and wind friction threshold for sand saltation over 
 the northern Africa. Finally, we summarize the dissertation and outlook future work in 
 Chapter 6. 
