% Chapter one

\chapter{Introduction}

\section{Background and Motivation}

 Mineral dust represents the second productive component of atmospheric aerosol 
 following after the sea-salt aerosol \citep{textor06}. Naturally, these mineral 
 particles are produced mainly by the aerolian wind ersion in arid and semiarid areas.     
 Anthropogenic sources of mineral aerosols include road dust and mineral dust due to land 
 changes by human activities [References check with my proposal]. Mineral aerosols can place
 important impacts on the Earth system through interations with atmospheric chemistry, 
 solar and terrestrial radiaton, clouds, and biosphere \citep{shao11}. 
 An accurate representation of dust cycle in the Earth climate model is thus critical 
 to assess these impacts. However, significant uncertainties prevail in quantifying 
 mineral dust sources due to poor understanding of dust uplifting mechanisms and the 
 lack of in situ measurements over the desert region. 

 Parameterization of processes such as saltation bombardment and sandblasting in a 
 chemistry transport model (CTM) requires knowledge of many parameters that are 
 poorly characterized, including surface wind speed, soil moisture, soil texture, 
 and surface state \citep{tegen94,ginoux01,zender03a}. Not surprisingly, recent 
 estimates in CTMs span from a few hundreds to over 4000 Tg for annual global 
 dust emissions \citep{huneeus11} and can vary by a factor as large as 10 at 
 regional scales for the same dust event(s) \citep{uno06}. An observation-based 
 approach, therefore, is needed to reduce these large uncertainties in estimate of 
 dust emissions and further improve the global modeling of atmospheric dust distribution 
 and their impacts.

\subsection{Impacts of dust aerosols}

Atmospheric aerosols play a crucial role in the global 
climate change. They affect earth energy budget directly 
by scattering and absorbing solar and terrestrial radiation, 
and indirectly through altering the cloud formation, 
lifetime, and radiative properties [Haywood and Boucher, 
2000; Ramanathan et al., 2001]. However, quantification of 
these effects in the current climate models is fraught with 
uncertainties. The global average of aerosol effective 
radiative forcing (ERF) were estimated to range from -0.1 
to -1.9 Wm2 with the best estimate of -0.9 Wm2 [Boucher et 
al., 2013], indicating that the cooling effects of aerosol 
might counteract the warming effects of 1.820.19 Wm2 caused
by the increase of carbon dioxide since the industrial 
revolution [Myhre et al., 2013]. The climate effects of 
aerosol particles depend on their geographical distribution, 
optical properties, and efficiency as cloud condensation 
nuclei (CCN). Key quantities pertain to the aerosol optical 
and cloud-forming properties include particle size 
distribution (PSD), chemical composition, mixing state, and 
morphology [Boucher et al., 2013]. While the daily aerosol 
optical depth (AOD) can be well measured from current 
satellite and ground-based remote sensing instrumentations 
[e.g., Holben et al., 1998; Kaufman et al., 2002], the 
accurate quantification of aerosol ERF is in no small part 
hindered by our limited knowledge about the aerosol PSD and 
refractive index (describing chemical composition and 
mixing state). To fully understand the role of aerosol 
particles in the global climate change, further development 
in observations along with retrieval algorithms for these 
aerosol microphysical properties from different platforms 
are thus highly needed [Mishchenko et al., 2004], and 
the focus of this two-part series study is the 
characterization of aerosol properties from ground-based 
passive remote sensing \citep{henze07}. 

\subsection{Parameterizations of dust emissions}

\subsection{Observations of dust aerosols}

 In the last decade, the in situ and satellite remote sensing observations 
 have greatly enhanced our understanding of the spatiotemporal variations
 of dust aerosols. 

\subsection{Recent inverse modeling studies for improving dust emission}

 In parallel with the advancement of in situ and remote sensing observations 
 of dust aerosols, techniques have been developed to use these observations 
 as constraints on dust sources.

\citet{koven08} have investigated ...

 Source function can be estimated based topography \citep{ginoux01}, 

\section{Main Goals of This Work}

 Based on the preceding discussions, this work aims at improved estimates of 
 global dust emissions through adjoint integration of AOD retrievals 
 from multiple satellite platforms (MODIS and MISR) with a CTM (GEOS-Chem). 
 The overall goal is to conduct the satellite-based global model estimates 
 of atmospheric dust distribution, and thereby advance the understanding of 
 the impacts of atmospheric mineral dust on climate change and air quality. 
 To accomplish this goal, this work pursues the following specific objectives:
 \begin{itemize}
 \item Develop a top-down numerical inversion scheme for constraining global dust emissions with a combined use of multi-platform AOD products and CTM adjoint, which also includes the sensitivity and error budget analysis for the optimization.
 \item Apply the inversion scheme developed in step 1 for a one year (i.e. 2008) of dust emissions with level 3 quality-controlled MODIS DB and MISR AOD products.
 \item A long-term (from 2001 to 2010) analysis of dust emissions will follow, along with studies on the seasonal and inter-annual variability of dust emissions, loadings, and direct radiative effects.
 \item Wherever possible, ground-based and field data will be used to validate and analyze the uncertainties of the inversion results.
 \end{itemize}
Although the adjoint optimization technique we use is similar to that in Dubovik et
al. [2008] and Yumimoto et al. [2007], this study differs from the those previous studies in that: (a) Multi-platform AOD products utilized to optimize dust emissions can provide tremendous dust information in fine spatial and temporal scales; (b) This study uses the satellite AOD retrievals only over and near dust source regions where dust has been transported a short distance with minimal influence of precipitation and anthropogenic
aerosols; (c) Optimization of the long-term dust emissions is conducted for every grid box as a function of time (e.g., on the weekly or month scale). Although the criteria for separation of a natural and anthropogenic dust source is not clear and sometime controversial in the literature [Denman et al., 2007], especially when considering the climatic feedback on dust emissions [Zhang et al., 2002], we believe that satellite-based optimization of global dust emissions in the last decade could improve our modeling of dust radiative forcing and potentially illuminate anthropogenic components of dust sources and loadings, currently estimated at 0-20\% though values as large as 50\% has been postulated [Ginoux et al., 2011; Tegen et al., 1996; 2004; Mahowald et al., 2004,].

\section{Organization of This Dissertation}

 We describe the GEOS-Chem simulation of mineral dust in Chapter 2 with emphersizing the 
 physical parameterization of dust sources, after which we present the implements for the 
 AOD observation operator and the adjoint capacity of dust emission within the GEOS-Chem
 ajoint model in Chapter 3. In chapter 4, we present a case study on optimizing the dust 
 emission estimates from the satellite (MODIS) radiances over the eastern Asia, 
 in which we also attempt to 
 simutaneously constrain the anthropogenic emissions of the \ce{SO2}, \ce{NO2}, \ce{NH3}, 
 and carboneous aerosols tegether with the dust aerosols. In chapter 5, we optimize the 
 dust source parameterization from multi-satellite AOD products, particularly in improving 
 the the estimates of soil erodibility and wind friction threshold for sand saltation over 
 the northern Africa. Finally, we summarize the dissertation and outlook future work in 
 Chapter 6. 
