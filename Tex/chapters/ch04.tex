%% 
\chapter{Optimizing Dust Source Parameterization} \label{chap:optdead}

\section{Introduction}

 Mineral dust accounts for about 40\% of the global annual emissions of tropospheric aerosols
 \citep{textor06,boucher13}. Naturally, these mineral
 particles are uplifited mainly by the aerolian wind ersion in arid and semiarid areas.
 Anthropogenic sources of mineral aerosols include road dust and mineral dust due to land
 changes by human activities \citep{tegen96,tegen04,ginoux12}.
 In the atmosphere, mineral dust can be transported thousands of miles away from its ources
 \citep{husar01,toon03} and influences many aspects of the Earth system through
 its interations with atmospheric chemistry,
 solar and terrestrial radiaton, clouds, and biosphere \citep{shao11,ravi11,satheesh05,carslaw10}. An accurate representation of dust cycle in the Earth climate model is thus critical
 to assess these impacts. However, significant uncertainties prevail in quantifying
 mineral dust sources due to the limited understanding of dust uplifting mechanisms and the
 lack of in situ measurements over the desert region.

 Parameterization of dust entrainments in a
 chemistry transport model (CTM) requires knowledge of many parameters that are
 poorly characterized, including surface wind speed, soil moisture, soil texture,
 and surface state and roughness \citep{tegen94,ginoux01,zender03a}.
 Not surprisingly, recent estimates of global annual dust emissions in CTMs
 span from a few hundreds to over 4000 Tg \citep{huneeus11} and can vary by
 a factor as large as fifteen at regional scales for the same dust event(s) \citep{uno06}.
 An observation-based approach, therefore, is needed
 to reduce these large uncertainties in estimate of
 dust emissions and further improve the global modeling of atmospheric dust distribution
 and their impacts.

 In the last decade, the \textit{in situ} and satellite remote sensing observations
 have greatly enhanced our understanding of the spatiotemporal variations and
 radiative effects of dust aerosol.
 This is especially true for satellite retrievals that are based on
 multi-spectral and multi-angle radiance measurements
 from NASA’s Earth Observing System satellite instruments, such as MODIS and MISR,
 which provide daily information of aerosol optical depth (AOD) across the globe
 \citep{king99,kaufman02,martonchik09}.
 The MODIS “deep blue” (DB) aerosol retrievals \citep{hsu06,hsu13}
 further made the aerosol information over bright deserts available.
 On the other hand, in situ observations (e.g. AERONET \citep{holben98})
 and experiment campaigns, even though sparse in their temporal and spatial coverage,
 usually have advantages in their accuracy,
 and thus are valuable for validating satellite aerosol products and model simulations.

 In this dissertation, I thus seek to contribute an improved understanding
 of the dust emissions through (i) \ldots, (ii) \ldots, and (iii).

\section{Necessary Implementations}

\subsection{Development of the wind speed distribution}

  In order to incorporate the variability of wind speed due to the subgrid scale
  circulations, we introduce a probability density function (PDF) of the wind speed
  within each grid box. The dust emission is computed according to the fraction of
  the PDF that exceeds the threshold value:
  \begin{equation}
  E_{d} = \int^\infty_{u_{*t}} E(u_*) p(u_*) d u_* \mbox{.}
  \end{equation}
  Where $E(u_*)$ is the emission as a function of the surface wind friction velocity,
  and $p(u_*)$ is the PDF of $u_*$ within the grid box.

  The PDF for surface wind speeds can be represented by a Weibull distribution
  [Justus et al., 1978] and has been used in recent studies [e.g., Grini and Zender,
  2004; Grini et al., 2005; Ridley et al., 2013] to charaterize the subgrid dust
  emissions. The PDF of a Weibull random variable $x$ is described by a shape factor $k$
  and a scale factor $c$:
  \begin{equation}
  p(x;c,k) = \frac{k}{c} (x/c )^{k-1}
         \exp{\left[ -(x/c)^k \right]} \mbox{,  for } x>0 \mbox{.} 
  \end{equation}
  One of the advantages in using the Weibull PDF is that it is analytically
  integrable with the cumulative distribution function:
  \begin{equation}
  P(x \leq x_1;c,k) = 1 - \exp{ \left[ -(x/c)^k\right] } \mbox{.}
  \end{equation}
  Based on above cumulative function, we cut off wind speeds with a minimum and a
  maximum wind speed to retain the central 98\% of the wind PDF. As a result, the
  lower and upper limits of wind speed are respectively:
  \begin{align} 
  x_{l} &= c \left[ -\ln{0.99} \right]^{\frac{1}{k}}  \\
  x_{u} &= c \left[ -\ln{0.01} \right]^{\frac{1}{k}} \label{eq:xu}
  \end{align}

  Parameters $k$ and $c$ can be estimated from the statistical mean $\bar{x}$ and
  variance $\sigma^2$ (of $x$), since they are related to $\bar{x}$ and $\sigma^2$:
  \begin{align}
  \bar{x}  &= c \Gamma(1+1/k)  \\
  \sigma^2 &= c^2 \left[ \Gamma(1+2/k) - \Gamma^2(1+1/k) \right]  
  \end{align}
  Where $\Gamma()$ is a gamma function. According to Justus et al. [1978], $k$ and $c$
  can be best estimated by:
  \begin{align}
  k &= (\sigma/\bar{x})^{-1.086} \\
  c &= \bar{x} \left[ \Gamma(1+1/k) \right]^{-1}
  \end{align}

  Thus, the only parameter that must be supplied beyound the mean wind speed is the
  variance ($\sigma^2$) of subgrid wind speeds within the grid box. Cakmur et al [2004]
  calculated the $\sigma^2$ by incorporating information from the parameterizations of
  the planetary boundary layer along with dry and moist convection. Here, we follow
  Grini and Zender [2004] and Grini et al [2005] that assumed an approximation of $k$
  based on Justus et al. [1978]:
  \begin{equation}
  k = 0.94u_*^{\frac{1}{2}}
  \end{equation}

  Finally, the dust emission flux is calculated by
  \begin{equation} \label{eq:Edp}
  E_{d,j} = A_m S \alpha \left( \displaystyle{\sum_{i=1}^3 M_{i,j}} \right)
          \frac{c_s \rho}{g}
          \displaystyle \int_{u_{*t}}^{u_{*u}} 
          u_*^b\left(1-\frac{u_{*t}}{u_*}\right)
          \left(1+\frac{u_{*t}}{u_*}\right)^2 
          p(u_*) d u_* \mbox{.}
  \end{equation}
  Where $u_{*u}$ is the upper limit of wind speed determined by equation (\ref{eq:xu}).

 \subsection{AOD observation operator}

  Inversion of dust emissions from AOD observations requires
  an observation operator that relates AOD to emission.
  Such observation operator of this study comprises
  a GEOS-Chem forward model that simulates aerosol concentrations
  with input of emissions and an AOD operator that computes AOD
  from aerosol concentrations.
  AOD at any wavelength is calculated from the sum of AODs
  of each component assuming external mixing state of aerosol
  particles:
  \begin{equation} \label{eq:aod} 
   \tau_\text{A} = \displaystyle \sum_{i=1}^{n_\text{spec}} 
                   {\beta_\text{ext}}_i c_i
                   \frac{3c_i {Q_\text{ext}}_i}{4 \rho_i, {r_\text{eff}}_i}
  \end{equation}
  where $n_\text{spec}$ is the number of aerosol components,
  $c_i$ is the aerosol mass concentration of the $i$th component,
  and ${\beta_\text{ext}}_i$ is the mass extinction efficiency.
  ${\beta_\text{ext}}_i$ is related to the

 \subsection{Adjoint for Dust Flux Parameterization}

   In simple, The dust emission flux considering subgrid wind speeds in equation
   (\ref{eq:Edp}) can be writen
   \begin{equation}
   E_{d,j} = C_j S^\prime \int^{u_{*u}}_{u_{*t}} Q_s(u_*,u_{*t},b) p(u_*) d u_* \mbox{,}
   \end{equation}
   where $S^\prime = S \alpha$, and $C_j=A_m \displaystyle \sum_{i=1}^3 M_{i,j}$. We combine
   the erodibility $S$ and sandblasting factor $\alpha$, because both of them not only are
   related to the soil texture but also describe the strength efficiency of dust emission.
   Given the the state of land surface and the properties of surface soil, the dust emission
   is a function of $S^\prime$, $b$, and $u_{*t}$.

   Here we implement the adjoint calculation for three parameters, i.e.,
   $S^\prime$, $b$, and $u_{*t}$. This implementation reqiures the partial
   derivatives of $E_{d,j}$ with respect to these parameters (when
   $u_* \geq u_{*t}$):
   \begingroup
   \allowdisplaybreaks
   \begin{align}
     \frac{\partial E_{d,j}}{\partial S^\prime} 
       &= \frac{E_{d,j}}{S^\prime}\mbox{,} \\
     \frac{\partial E_{d,j}}{\partial b} 
       &= C_j S^\prime \int^{u_{*u}}_{u_{*t}} 
          \frac{\partial Q_s}{\partial b} p(u_*) d u_* \mbox{,} \label{eq:dEdb}\\
     \frac{\partial E_{d,j}}{\partial u_{*t}}
       & = C_j S^\prime \int^{u_{*u}}_{u_{*t}} 
          \frac{\partial Q_s}{\partial u_{*t}} p(u_*) d u_* \mbox{.} \label{eq:dEdut}
   \end{align}
   \endgroup
   These gradients of $Q_s$ in equations (\ref{eq:dEdb} and \ref{eq:dEdut}) can be
   calculated by
   \begingroup
   \allowdisplaybreaks
   \begin{align}
   \frac{\partial Q_s}{\partial b} &= Q_s(u_*,u_{*t},b) \ln{u_*} \\
   \frac{\partial Q_s}{\partial u_{*t}} &=\frac{c_s \rho}{g} u_*^b
         \left[\frac{1}{u_*} - \frac{2u_{*t}}{u_*^2} 
         - \frac{3u_{*t}^2}{u_*^3}\right] 
   \end{align}
   \endgroup

\section{Constraints from Multi-Satellite Observations}

\section{Experiment Design}

\section{Constrained Dust Emission Scheme}

\section{Validations}

\section{Summary}
