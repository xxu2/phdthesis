% Chapter 3
\chapter{Adjoint Inversion of Dust Emissions}

\section{The Inversion Framework}

 Let $\mathbf{x}$ denote a state vector of $n$ parameters to be constrained 
 and $\mathbf{y}$ an observation vector assembled by $m$ measurements, 
 and let $\mathbf{F}$ indicate a forward model that describes the physics of
 the measurement process. Then, we can express the relationship between
 the observation vector and the state vector as
 \begin{equation}
 \mathbf{y} = \mathbf{F}(\mathbf{x}) + \boldsymbol{\epsilon} \mbox{,}
 \end{equation}
 where $\boldsymbol{\epsilon}$ is an experimental error term that includes
 observation noise and forward modeling uncertainty. 

 For this work, the observation vector $\mathbf{y}$ comprises aerosol mass 
 cocentrations or aerosol optical depth (AOD).  

\section{GEOS-Chem Adjoint}

 The GES-Chem adjoint model was specifically developed for \ldots \citep{henze07,henze09}.
 It has been widely used to \ldots [references\ldots]. 

 It caluclates the adjoint, or the transpose of Jacobian matrix of receptor with respect 
 to the state vector, following the \ldots

 Based on the infrastructure of GEOS-Chem, we need to develope (1) an observation operator 
 that maps the aerosol concentration into the observation space, (2) the capacity of 
 calculating the adjoint with respect to dust emission flux, and (3) the capacity of 
 calculating the adjoint with respect to parameters in dust emission scheme. 

\section{Implements of AOD observation operator}

Two types of observation operator

\section{Implements of Adjoint for Dust Emissions}
\section{Implements of Adjoint for Dust Flux Parameterization}

   In simple, The dust emission flux considering subgrid wind speeds in equation 
   (\ref{eq:Edp}) can be writen
   \begin{equation}
   E_{d,j} = C_j S^\prime \int^{u_{*u}}_{u_{*t}} Q_s(u_*,u_{*t},b) p(u_*) d u_* \mbox{,}
   \end{equation}
   where $S^\prime = S \alpha$, and $C_j=A_m \displaystyle \sum_{i=1}^3 M_{i,j}$. We combine
   the erodibility $S$ and sandblasting factor $\alpha$, because both of them not only are 
   related to the soil texture but also describe the strength efficiency of dust emission. 
   Given the the state of land surface and the properties of surface soil, the dust emission
   is a function of $S^\prime$, $b$, and $u_{*t}$. 

   Here we implement the adjoint calculation for three parameters, i.e., 
   $S^\prime$, $b$, and $u_{*t}$. This implementation reqiures the partial
   derivatives of $E_{d,j}$ with respect to these parameters (when 
   $u_* \geq u_{*t}$): 
   \begingroup
   \allowdisplaybreaks
   \begin{align}
     \frac{\partial E_{d,j}}{\partial S^\prime} 
       &= \frac{E_{d,j}}{S^\prime}\mbox{,} \\
     \frac{\partial E_{d,j}}{\partial b} 
       &= C_j S^\prime \int^{u_{*u}}_{u_{*t}} 
          \frac{\partial Q_s}{\partial b} p(u_*) d u_* \mbox{,} \label{eq:dEdb}\\
     \frac{\partial E_{d,j}}{\partial u_{*t}}
       & = C_j S^\prime \int^{u_{*u}}_{u_{*t}} 
          \frac{\partial Q_s}{\partial u_{*t}} p(u_*) d u_* \mbox{.} \label{eq:dEdut}
   \end{align}
   \endgroup
   These gradients of $Q_s$ in equations (\ref{eq:dEdb} and \ref{eq:dEdut}) can be 
   calculated by
   \begingroup
   \allowdisplaybreaks
   \begin{align}
   \frac{\partial Q_s}{\partial b} &= Q_s(u_*,u_{*t},b) \ln{u_*} \\
   \frac{\partial Q_s}{\partial u_{*t}} &=\frac{c_s \rho}{g} u_*^b
         \left[\frac{1}{u_*} - \frac{2u_{*t}}{u_*^2} 
         - \frac{3u_{*t}^2}{u_*^3}\right] 
   \end{align}
   \endgroup
