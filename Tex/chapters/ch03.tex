% Chapter 3
\chapter{Adjoint Inversion of Dust Emissions}

 \textit{Overview} \hspace{0.2cm}

\section{The Adjoint Inversion Framework}

 The adjoint of the GEOS-Chem model was developed specifically 
 for inverse modeling of aerosol (or their precursors) 
 and gas emissions \citep{henze07,henze09}, 
 and it is continuously improved and maintained 
 by the GEOS-Chem Adjoint and Data Assimilation 
 Working Group and its users 
 (\url{http://wiki.seas.harvard.edu/geos-chem/index.php/GEOS-Chem_Adjoint}). 
 The strength of the adjoint model is 
 its ability to efficiently calculate model sensitivities 
 with respect to large sets of model parameters, 
 such as aerosol emissions at each grid box. 
 These sensitivities can serve as the gradients needed for 
 inverse modeling of aerosol emissions. 
 Recent studies have used the GEOS-Chem adjoint with 
 satellite observations to constrain sources of species 
 such as \ce{CO} \citep{kopacz09,kopacz10,jiang11}, 
 \ce{CH4} \citep{,wecht12}, and \ce{O3} \citep{parrington12}
 to diagnose source regions for long-range transport \citep{henze09,kopacz11}, 
 and to provide guidance on future geostationary observations 
 of surface air quality \citep{zoogman11}.

 \subsection{Inversion strategy}

 Let $\mathbf{x}$ denote a state vector of $n$ parameters to be constrained 
 and $\mathbf{y}$ an observation vector assembled by $m$ measurements, 
 and let $\mathbf{F}$ indicate a forward model that describes the physics of
 the measurement process. Then, we can express the relationship between
 the observation vector and the state vector as
 \begin{equation} \label{eq:fx}
 \mathbf{y} = \mathbf{F}(\mathbf{x}) + \bdeps \mbox{,}
 \end{equation}
 where $\bdeps$ is an experimental error term that includes
 observation noise and forward modeling uncertainty. 

 For this work, the observation 
 vector $\mathbf{y}$ comprises measurements of 
 aerosol loading, such as mass concentrations or optical depth, at any temporal 
 and spatial scale. The components of the sate vector $\mathbf{x}$ could vary 
 according to our inversion focus. For the inversion of aerosol emission estimats 
 (as in the Chapter \ref{chap:optems}), 
 $\mathbf{x}$ comprises the emission fluxes (or their scaling factors) of defined 
 aerosol species within each grid cell of specified temporal resolution. 
 In constrast, $\mathbf{x}$ consists of 
 dust emitting parameters (or their scaling factors) when we tend to constrain 
 the dust emission parameterization (as in the Chapter \ref{chap:optdead}). 
 The forward model $\mathbf{F}$ 
 represents the GEOS-Chem that maps parameters from the state space to the 
 observation space. The inversion of the state vector from these measurements is often 
 an ill-posed problem due to non-linearity and limited sensitivity of these observed 
 quantities to the constrained parameters. We need to combine additional constraints 
 to make the problem amenable to inversion. 

 \textit{A propri} information describes our knowledge of the state vector before measurements
 are applied. \textit{A propri} constraint is commonly used to achieve a well-defined stable
 and physically reasonable solution to an ill-posed problem. Usually, \textit{a propri} 
 knowledge comprises both a mean state $\bdxa$ and its error
 $\bdeps_\text{a}$:
 \begin{equation} \label{eq:xa}
  \mathbf{x} = \bdxa + \bdeps_\text{a}
 \end{equation} 

 Under assumption of Gaussian-distributed errors, the Maximum A Posteriori solution of
  equations \eqref{eq:fx} and \eqref{eq:xa} according to the Bayesian approach
 corresponds to the state vector that minimizes the quadratic cost function 
 \citep{rodgers00}: 
 \begin{equation} \label{eq:cost}
 J(\mathbf{x}) = \frac{1}{2} \left[ \mathbf{F(x)-y}\right]^T
                 \bdsy^{-1} \left[ \mathbf{F(x)-y}\right]
               + \frac{1}{2} \gamma \left( \mathbf{x}-\bdxa \right)
                 \bdsa^{-1} \left( \mathbf{x}-\bdxa \right) \mbox{,}
 \end{equation}
 where $T$ indicates the transpose operation, $\bdsy$ is the error covariance matrix 
 of measurements, $\bdsa$ is the error covariance matrix of the \textit{a priori}, 
 and $\gamma$ is the regularization paramter. These two terms on the right side of 
 equation \eqref{eq:cost} respresent the total sqaured fitting error incurred owing 
 to departures of model predictions from the observations and the penalty error
 incurred owing to depatures of the estimates from the \textit{a priori}, respectively.
 Thus, the minimization of $J(\mathbf{x})$ achieves the objectives of improving the 
 agreement between the model and the measurements while ensuring that the solution 
 remains within a reasonable range and degree of smoothness. 

 The regularization parameter $\gamma$ in the calculation of $J(\mathbf{x})$ acts weights
 to balance the fitting error and the penalty error. Clearly, a good assignment of 
 $\gamma$ is of crucial importance for the statistically optimal solution. High values 
 of $\gamma$ can lead to over-smoothing of the solution with little improvement to the
 fitting residuals, while low values minimize the error term at the cost of greatly 
 increasing the penalty term. Optimal values of $\gamma$ can be identified at the 
 corner of the so-called L-curve \citep{hansen98}.

 In principle, solving this inverse problem is tantamount to a pure 
 mathematical minimization procedure. The minimization of $J(\mathbf{x})$ 
 is performed with an iterative quasi-Newton optimization approach 
 using the L-BFGS-B algorithm [Byrd et al., 1995; Zhu et al., 1994], 
 which offers bounded minimization to ensure the solution stays 
 within a physically reasonable range. The L-BFGS-B algorithm 
 requires knowledge of $\mathbf{x}$ and $J(\mathbf{x})$, 
 as well as the gradient of $J(\mathbf{x})$ with respect to 
 $\mathbf{x}$, $\Delta_\mathbf{x}J$. By linearizing the forward model F(x), 
 we can determine $\Delta_\mathbf{x}J$ by
 \begin{equation} \label{eq:dcost}
  \Delta_\mathbf{x}J = \mathbf{K}^T \bdsy^{-1} \left[ \mathbf{F(x)-y}\right]
                     + \gamma \bdsa^{-1} \left( \mathbf{x}-\bdxa \right) \mbox{,} 
 \end{equation}
 where $\mathbf{K}$ is the Jacobian matrix of $\mathbf{F(x)}$ 
 with respect to $\mathbf{x}$, which is computed analytically by adjoint method 
 in the GEOS-Chem adjoint. At each iteration, improved estimates 
 of the state vector are implemented and the forward simulation is recalculated. 
 The convergence criterion to determine the optimal solution is the smallness 
 of the $J(\mathbf{x})$ reduction and the norm of $\Delta_\mathbf{x}J$. 
 The iteration stops when the reduction of $J(\mathbf{x})$ is less than 1\% within 
 five continuous iterations. Then, the optimal solutions are identified 
 corresponding to the smallest norm of $\Delta_\mathbf{x}J$ among 
 these five last iterations. 

 \subsection{GEOS-Chem adjoint modeling}

 The GES-Chem adjoint model was specifically developed for \ldots \citep{henze07,henze09}.
 It has been widely used to \ldots [references\ldots]. 

 It caluclates the adjoint, or the transpose of Jacobian matrix of receptor with respect 
 to the state vector, following the \ldots

 Based on the infrastructure of GEOS-Chem, we need to develope (1) an observation operator 
 that maps the aerosol concentration into the observation space, (2) the capacity of 
 calculating the adjoint with respect to dust emission flux, and (3) the capacity of 
 calculating the adjoint with respect to parameters in dust emission scheme. 

\section{Implements of AOD observation operator}

Two types of observation operator

\section{Implements of Adjoint for Dust Emissions}
\section{Implements of Adjoint for Dust Flux Parameterization}

   In simple, The dust emission flux considering subgrid wind speeds in equation 
   (\ref{eq:Edp}) can be writen
   \begin{equation}
   E_{d,j} = C_j S^\prime \int^{u_{*u}}_{u_{*t}} Q_s(u_*,u_{*t},b) p(u_*) d u_* \mbox{,}
   \end{equation}
   where $S^\prime = S \alpha$, and $C_j=A_m \displaystyle \sum_{i=1}^3 M_{i,j}$. We combine
   the erodibility $S$ and sandblasting factor $\alpha$, because both of them not only are 
   related to the soil texture but also describe the strength efficiency of dust emission. 
   Given the the state of land surface and the properties of surface soil, the dust emission
   is a function of $S^\prime$, $b$, and $u_{*t}$. 

   Here we implement the adjoint calculation for three parameters, i.e., 
   $S^\prime$, $b$, and $u_{*t}$. This implementation reqiures the partial
   derivatives of $E_{d,j}$ with respect to these parameters (when 
   $u_* \geq u_{*t}$): 
   \begingroup
   \allowdisplaybreaks
   \begin{align}
     \frac{\partial E_{d,j}}{\partial S^\prime} 
       &= \frac{E_{d,j}}{S^\prime}\mbox{,} \\
     \frac{\partial E_{d,j}}{\partial b} 
       &= C_j S^\prime \int^{u_{*u}}_{u_{*t}} 
          \frac{\partial Q_s}{\partial b} p(u_*) d u_* \mbox{,} \label{eq:dEdb}\\
     \frac{\partial E_{d,j}}{\partial u_{*t}}
       & = C_j S^\prime \int^{u_{*u}}_{u_{*t}} 
          \frac{\partial Q_s}{\partial u_{*t}} p(u_*) d u_* \mbox{.} \label{eq:dEdut}
   \end{align}
   \endgroup
   These gradients of $Q_s$ in equations (\ref{eq:dEdb} and \ref{eq:dEdut}) can be 
   calculated by
   \begingroup
   \allowdisplaybreaks
   \begin{align}
   \frac{\partial Q_s}{\partial b} &= Q_s(u_*,u_{*t},b) \ln{u_*} \\
   \frac{\partial Q_s}{\partial u_{*t}} &=\frac{c_s \rho}{g} u_*^b
         \left[\frac{1}{u_*} - \frac{2u_{*t}}{u_*^2} 
         - \frac{3u_{*t}^2}{u_*^3}\right] 
   \end{align}
   \endgroup
