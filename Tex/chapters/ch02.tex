% Chapter 2
\chapter{Top-Down Inversion Framework}

\section{Inversion Theory}

\section{GEOS-Chem Forward Modeling}

The GEOS-Chem chemistry transport model is used to simulate the ambient concentrations 
of atmospheric aerosols.

The GEOS-Chem aerosol simulatiion was based on the GOCART model [Chin et al., 2002], 
particularly for wet scavenging, with updates described by Park et al. [2004]. 

\section{GEOS-Chem Adjoint Modeling}



 \textit{Overview} \hspace{0.2cm} 
 This chapter presents how the physical process that mineral dust 
 involved are quantitatively respresented in a chemistry transport model, i.e., 
 the GEOS-Chem model. These processes include the uplifting of dust from soil surface, 
 the transport within the atmosphere, and despoistion of dust to the surface. 

\section{Modeling of Dust Emission}

\subsection{Physical parameterization of the dust emission}

  The dust emission, aerolian wind erosion that results in production of mineral aerosols
  from soil grains, involves complex and nonlinear processes that are governed by the 
  meteorology as well as by the state and properties of the land surfaces. Laboratory 
  [Iversen and White, 1982] and field [Shao et al., 1996; Zender et al., 2003] wind tunnel
  studies suggested that dust is primiarily injected into the atmosphere during the 
  sandblasting caused by the saltation bombardment [Alfaro and Gomes, 2001; Grini et al.,
  2002]. The clay- and silt-sized soil particles have strong inter-cohesive force... 
  The saltation of sand-sized particles ... requires least threshold of wind speed...


  The most important factors include wind friction velocity and its threshold for 
  saltation, vegetation cover, soil minerology, and surface soil moisture. 

  In this study, the physical parameterization of dust emission is taken from a Dust 
  Entrainment and Deposition (DEAD) model developed by Zender et al [2003a]. The DEAD scheme
  calculates the wind friction threshold ($u_{*t}$) as a function of the Reynolds number 
  following Iversen and White [1982] and Marticorena and Bergametti [1995]. Three processes 
  are also considered to modify the $u_{*t}$: the drag partitioning owing to the momentum 
  captured by nonerodible roughness elements, the Owen effect, and moisture inhition. The 
  horizontal saltation flux ($Q_s$) that is defined as the vertical integral of the 
  stream-wise soil flux density is calculated following the theory of White [1979]:
  \begin{equation}
  Q_s(u_*,u_{*t}) = \frac{c_s \rho}{g} u_*^3
        \left(1-\frac{u_{*t}}{u_*}\right)
        \left(1+\frac{u_{*t}}{u_*}\right)^2 \mbox{,}
  \end{equation}
  where, $c_s=2.61$, $\rho$ is the air density at surface level, and $u*$ is the wind 
  friction velocity. Thus, it assumes the saltaion flux is quasi-lienarly the $u_*^3$ when
  $u_*$ exceeds the $u_{*t}$. It also neglect the dependence of total $Q_s$ on the soil size. 

  the total
   vertical mass flux of dust into transport bin $j$ is
   \begin{equation} \label{eq:Ed}
   E_{d,j} = 
     \begin{cases} T_0 A_m S \alpha Q_s
                   \displaystyle \sum_{i=1}^3 M_{i,j} & \mbox{if $u_* \geq u_{*t}$,} \\
                   0 & \mbox{if $u_* < u_{*t}$,}
     \end{cases}
   \end{equation}
   where, $T_0$ is a tuning factor chosen to adjust the global amount, $A_m$ is the fraction
   of bare solil exposed in a model grid cell, $S$ is called "erodiblity" or "perferential 
   source function", $\alpha$ is the sandblasting mass efficiency factor which depends on the
   mass fraction of clay particles in the parent soil, and $M_{i,j}$ indicates the mass 
   fraction of \textit{i}th source mode carried into the \textit{j}th transport mode.

\subsection{Development of the wind speed distribution}

  In order to incorporate the variability of wind speed due to the subgrid scale
  circulations, we introduce a probability density function (PDF) of the wind speed
  within each grid box. The dust emission is computed according to the fraction of 
  the PDF that exceeds the threshold value:
  \begin{equation}
  E_{d} = \int^\infty_{u_{*t}} E(u_*) p(u_*) d u_* \mbox{.}
  \end{equation}
  Where $E(u_*)$ is the emission as a function of the surface wind friction velocity,
  and $p(u_*)$ is the PDF of $u_*$ within the grid box. 

  The PDF for surface wind speeds can be represented by a Weibull distribution 
  [Justus et al., 1978] and has been used in recent studies [e.g., Grini and Zender,
  2004; Grini et al., 2005; Ridley et al., 2013] to charaterize the subgrid dust 
  emissions. The PDF of a Weibull random variable $x$ is described by a shape factor $k$ 
  and a scale factor $c$:
  \begin{equation}
  p(x;c,k) = \frac{k}{c} (x/c )^{k-1}
         \exp{\left[ -(x/c)^k \right]} \mbox{,  for } x>0 \mbox{.} 
  \end{equation}
  One of the advantages in using the Weibull PDF is that it is analytically 
  integrable with the cumulative distribution function:
  \begin{equation}
  P(x \leq x_1;c,k) = 1 - \exp{ \left[ -(x/c)^k\right] } \mbox{.}
  \end{equation} 
  Based on above cumulative function, we cut off wind speeds with a minimum and a 
  maximum wind speed to retain the central 98\% of the wind PDF. As a result, the 
  lower and upper limits of wind speed are respectively:
  \begin{align} 
  x_{l} &= c \left[ -\ln{0.99} \right]^{\frac{1}{k}}  \\
  x_{u} &= c \left[ -\ln{0.01} \right]^{\frac{1}{k}} \label{eq:xu}
  \end{align}

  Parameters $k$ and $c$ can be estimated from the statistical mean $\bar{x}$ and 
  variance $\sigma^2$ (of $x$), since they are related to $\bar{x}$ and $\sigma^2$: 
  \begin{align}
  \bar{x}  &= c \Gamma(1+1/k)  \\
  \sigma^2 &= c^2 \left[ \Gamma(1+2/k) - \Gamma^2(1+1/k) \right]  
  \end{align}
  Where $\Gamma()$ is a gamma function. According to Justus et al. [1978], $k$ and $c$ 
  can be best estimated by:
  \begin{align}
  k &= (\sigma/\bar{x})^{-1.086} \\
  c &= \bar{x} \left[ \Gamma(1+1/k) \right]^{-1}
  \end{align}
  
  Thus, the only parameter that must be supplied beyound the mean wind speed is the 
  variance ($\sigma^2$) of subgrid wind speeds within the grid box. Cakmur et al [2004] 
  calculated the $\sigma^2$ by incorporating information from the parameterizations of 
  the planetary boundary layer along with dry and moist convection. Here, we follow 
  Grini and Zender [2004] and Grini et al [2005] that assumed an approximation of $k$ 
  based on Justus et al. [1978]:
  \begin{equation}
  k = 0.94u_*^{\frac{1}{2}}
  \end{equation}

  Finally, the dust emission flux is calculated by
  \begin{equation} \label{eq:Edp}
  E_{d,j} = A_m S \alpha \left( \displaystyle{\sum_{i=1}^3 M_{i,j}} \right)
          \frac{c_s \rho}{g}
          \displaystyle \int_{u_{*t}}^{u_{*u}} 
          u_*^b\left(1-\frac{u_{*t}}{u_*}\right)
          \left(1+\frac{u_{*t}}{u_*}\right)^2 
          p(u_*) d u_* \mbox{.}
  \end{equation}  
  Where $u_{*u}$ is the upper limit of wind speed determined by equation (\ref{eq:xu}).
  

\section{Modeling of Dust Transport and Deposition}

